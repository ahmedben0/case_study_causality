% !TEX encoding = UTF-8 Unicode
\documentclass{article}
\usepackage[utf8]{inputenc}
\usepackage{geometry}
\geometry{letterpaper}
\usepackage[parfill]{parskip}
\usepackage{graphicx}

\usepackage[numbers,sort&compress]{natbib}
\usepackage{amssymb}
\usepackage{amsmath}
\usepackage[english]{babel}   %    
\usepackage[T1]{fontenc}
\usepackage[autolanguage]{numprint}
%% \usepackage[latin1]{inputenc}
%% \usepackage[cyr]{aeguill}
\usepackage{color}


\usepackage{hyperref}
\usepackage{listings}

\usepackage{tabto}
\usepackage{array}
\usepackage{amsmath}

\usepackage{hyperref}

\usepackage{url}
\usepackage{wrapfig}

\usepackage{amssymb}

\usepackage{caption}

\definecolor{backcolour}{rgb}{0.97,0.95,0.93}

\lstdefinestyle{mystyle}{
    backgroundcolor=\color{backcolour},
}

\author{\Large \textsc{Mohammed FELLAJI, Ahmed BEN AISSA}}
\date{September, 2020}

\begin{document}

\hypersetup{pdfborder=0 0 0} 		%pour enlever le cadre rouge dans la tables des matières


\makeatletter
  \begin{titlepage}
  \centering
     {\large \textsc{   }}\\
     \vspace{1em}
    \centering
      \includegraphics[width=0.5 \textwidth]{figures/LogoCS.png} \\
    \vspace{4cm}
      {\LARGE\textbf{Inférence Statistique des Relations de Causalité}\\
       \vspace{1em}
       {\large\textbf{
       \textit{\LARGE{Vers des algorithmes d'apprentissage éthique?}}}}\\
    \vspace{4cm}
    \centering
     {\Large \textsc{Encadrant : Frédéric Pennerath}}\\
     \vspace{1em}
        {\Large \@author} \\
        \vspace{3em}
        {\Large \@date} }\\
  \end{titlepage}
 
 
\makeatother

\tableofcontents
%% \listoffigures

%% \newpage
%% \listoftables



%%%%%%%%%%%%%%%
\newpage
\section{Introduction}


\newpage
\section{Bibliographic study}
one of the main papers : \cite{pearl2019seven} \href{https://cacm.acm.org/magazines/2019/3/234929-the-seven-tools-of-causal-inference-with-reflections-on-machine-learning/fulltext?mobile=false}{link}

\subsection{let's start with an example}

\subsection{Causation vs Association}
\cite{pearl2019seven} \cite{hernan2020causal}  \cite{pearl2010mathematics} One of commun phrases in statistics is 

\subsection{Association - Intervention - Counterfactuals}

 \cite{pearl2019seven} 

\subsection{Paper : Judea Pearl}

see \cite{pearl2010mathematics} \\
One of the best paper so far. It explains clearly the difference between the causal concept and the associational concept (for example : correlation). The first concept ...


%% http://singapore.cs.ucla.edu/csl_papers.html     http://bayes.cs.ucla.edu/home.htm


\subsection{Notations}
see \cite{yao2020survey} and \cite{hernan2020causal}

\subsection{definition of counterfactual}

\subsection{Simpson's paradoxe}

- give definition :

- give example : Cholesterol  

\subsection{Average Causal Effect}

see \cite{hernan2020causal}



\newpage
\section{Definition of the causality}

Perhaps the most important message of the discussion and methods presented in this paper would be a widespread awareness that (1) all studies concerning causal relations must begin with causal assumptions of some sort and (2) that a friendly and formal language is currently available for articulating such assumptions.\cite{pearl2010mathematics}

\cite{rubin2005causal}

\section{Python Library : DoWhy}
library developed by Microsoft \cite{dowhy}, blog article available \href{https://www.microsoft.com/en-us/research/blog/dowhy-a-library-for-causal-inference/}{link}


\section{Causal discovery}

A traditional way to discover causal relations is to use interventions or randomized experiments, which is, however, in many cases of interest too expensive, too time- consuming, unethical, or even impossible. Therefore, inferring the underlying causal structure from purely observational data, or from combinations of observational and experimental data, has drawn much attention in various disciplines. With the rapid accumulation of huge volumes of data, it is necessary to develop automatic causal search algorithms that scale well.\cite{10.3389/fgene.2019.00524}

\newpage
\bibliographystyle{ieeetr} % plain : no order -- ieeetr : sorted
\nocite{*}    % print all references
\bibliography{reference/ref}

\end{document}
