% !TEX encoding = UTF-8 Unicode
\documentclass{article}
\usepackage[utf8]{inputenc}
\usepackage{geometry}
\geometry{letterpaper}
\usepackage[parfill]{parskip}
\usepackage{graphicx}

\usepackage[numbers,sort&compress]{natbib}
\usepackage{amssymb}
\usepackage{amsmath}
\usepackage[english]{babel}   %    
\usepackage[T1]{fontenc}
\usepackage[autolanguage]{numprint}
%% \usepackage[latin1]{inputenc}
%% \usepackage[cyr]{aeguill}
\usepackage{color}


\usepackage{hyperref}
\usepackage{listings}

\usepackage{tabto}
\usepackage{array}
\usepackage{amsmath}

\usepackage{hyperref}

\usepackage{url}
\usepackage{wrapfig}

\usepackage{amssymb}

\usepackage{caption}

\definecolor{backcolour}{rgb}{0.97,0.95,0.93}

\lstdefinestyle{mystyle}{
    backgroundcolor=\color{backcolour},
}

\author{\Large \textsc{Mohammed FELLAJI, Ahmed BEN AISSA}}
\date{September, 2020}

\begin{document}

\hypersetup{pdfborder=0 0 0} 		%pour enlever le cadre rouge dans la tables des matières


\makeatletter
  \begin{titlepage}
  \centering
     {\large \textsc{   }}\\
     \vspace{1em}
    \centering
      \includegraphics[width=0.5 \textwidth]{figures/LogoCS.png} \\
    \vspace{4cm}
      {\LARGE\textbf{Inférence Statistique des Relations de Causalité}\\
       \vspace{1em}
       {\large\textbf{
       \textit{\LARGE{Vers des algorithmes d'apprentissage éthique?}}}}\\
    \vspace{4cm}
    \centering
     {\Large \textsc{Encadrant : Frédéric Pennerath}}\\
     \vspace{1em}
        {\Large \@author} \\
        \vspace{3em}
        {\Large \@date} }\\
  \end{titlepage}
 
 
\makeatother

\tableofcontents
%% \listoffigures

%% \newpage
%% \listoftables



%%%%%%%%%%%%%%%
\newpage
\section{Introduction}


\newpage
\section{Bibliographic study}

\subsection{let's start with an example}

\subsection{Causation vs Association}

\subsection{Paper : Judea Pearl}

see \cite{pearl2010mathematics} \\
One of the best paper so far. It explains clearly the difference between the causal concept and the associational concept (for example : correlation). The first concept ...

\subsection{Notations}
see \cite{yao2020survey} and \cite{hernan2020causal}

\subsection{definition of counterfactual}

\subsection{Simpson's paradoxe}

\subsection{Average Causal Effect}

see \cite{hernan2020causal}



\newpage
\section{Definition of the causality}

Perhaps the most important message of the discussion and methods presented in this paper would be a widespread awareness that (1) all studies concerning causal relations must begin with causal assumptions of some sort and (2) that a friendly and formal language is currently available for articulating such assumptions.\cite{pearl2010mathematics}

\cite{rubin2005causal}

\newpage
\bibliographystyle{ieeetr} % plain : no order -- ieeetr : sorted
\nocite{*}    % print all references
\bibliography{reference/ref}

\end{document}
